LLVM具有非常灵活的架构,可以添加一个新的目标后端。后端的核心是目标描述,大部分代码都是从目标描述中生成的。本章中,将学习如何添加对历史CPU的支持。

本章中,将学习以下主题:

\begin{itemize}
\item
为创建一个新的后端做准备,介绍M88k CPU架构,了解在哪里可以找到所需的信息

\item
将新架构添加到Triple类中,了解如何使LLVM意识到新的CPU架构

\item
LLVM中扩展ELF文件格式定义,向处理ELF对象文件的库和工具中添加对特定于M88k的支持

\item
了解创建目标描述将应用TableGen语言的知识,对目标描述中的注册文件和指令进行建模

\item
将M88k后端添加到LLVM,了解LLVM后端所需的最小基础结构

\item
实现汇编器解析器,了解如何开发汇编器

\item
创建反汇编器,了解如何创建反汇编器
\end{itemize}

本章结束时,将了解如何为LLVM添加一个新的后端。将了解在目标描述中开发寄存器文件定义和指令定义的知识,并了解如何根据描述创建汇编程序和反汇编程序。























