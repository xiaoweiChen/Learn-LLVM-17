In this chapter, you added two different instruction selections to your backend: instruction selection via the selection DAG, and global instruction selection. For this, you had to define the calling convention in the target description. In addition, you needed to implement register and instruction information classes, which give you access to information generated from the target description but which you also needed to enhance with additional information. You learned that the stack frame layout and prolog generation are needed later. To translate an example, you added a class to emit machine instructions, and you created the configuration of the backend. You also learned how global instruction selection works. Finally, you gained some guidance on how you can develop the backend on your own.

In the next chapter, we will look at some tasks that can be done after instruction selection – we will add a new pass in the pipeline of the backend, look at how to integrate the backend into the clang compiler, and how to cross-compile to a different architecture.

