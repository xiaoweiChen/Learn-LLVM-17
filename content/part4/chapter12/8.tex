使用本章和前一章的代码,创建了一个后端,可以将一些LLVM IR转换为机器码。看到后端能够正常工作是非常令人满意的,但不能用于实际任务。需要编写更多的代码。以下是如何进一步发展后端的秘诀:

\begin{itemize}
\item
第一个决定是,是使用GlobalISel,还是选择DAG。根据我们的经验,GlobalISel更容易理解和开发,但是LLVM源代码树中的所有目标都实现了选择DAG,并且开发者可能已经有了使用它的经验。

\item
接下来,应该定义用于加减整数值的指令,这与按位和指令类似。

\item
之后,应该实现加载和存储指令。由于需要转换不同的寻址模式,这就更复杂了。最可能的情况是,需要处理索引,对数组中的一个元素进行寻址,这很可能需要前面定义的加法指令。

\item
最后,可以完全实现帧和调用的向下转译。此时,可以翻译一个简单的“Hello, world!”样式的应用程序变成可运行的程序。

\item
下一个逻辑步骤是实现分支指令,支持循环的转换。为了生成最优的代码,需要在指令信息类中实现分支分析方法。
\end{itemize}

这时,自定义后端已经可以翻译简单的算法。还应该获得足够的经验,可以根据优先级开发缺失的部分。

















