A real compiler has to deal with many files. Usually, the developer calls the compiler with the name of the main compilation unit. This compilation unit can refer to other files – for example, via \#include directives in C or import statements in Python or Modula-2. An imported module can import other modules, and so on. All these files must be loaded into memory and run through the analysis stages of the compiler. During development, a developer may make syntactical or semantical errors. When detected, an error message, including the source line and a marker, should be printed. This essential component is not trivial.

Luckily, LLVM comes with a solution: the llvm::SourceMgr class. A new source file is added to SourceMgr with a call to the AddNewSourceBuffer() method. Alternatively, a file can be loaded with a call to the AddIncludeFile() method. Both methods return an ID to identify the buffer. You can use this ID to retrieve a pointer to the memory buffer of the associated file. To define a location in the file, you can use the llvm::SMLoc class. This class encapsulates a pointer to the buffer. Various PrintMessage() methods allow you to emit errors and other informational messages to the user.