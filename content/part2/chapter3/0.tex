As we learned in the previous chapter, a compiler is typically divided into two parts – the frontend and the backend. In this chapter, we will implement the frontend of a programming language – that is, the part that mainly deals with the source language. We will learn about the techniques that realworld compilers use and apply them to our programming languages.

Our journey will begin with us defining our programming language’s grammar and end with an abstract syntax tree (AST), which will become the base for code generation. You can use this approach for every programming language for which you would like to implement a compiler.

In this chapter, you will learn about the following:

\begin{itemize}
\item
Defining a real programming language, where you will learn about the tinylang language, which is a subset of a real programming language, and for which you will implement a compiler frontend

\item
Organizing the directory structure of a compiler project

\item
Knowing how to handle multiple input files for the compiler

\item
The skill of handling user messages and informing them of issues in a pleasant manner

\item
Building the lexer using modular pieces

\item
Constructing a recursive descent parser from the rules derived from a grammar to perform syntax analysis

\item
Performing semantic analysis by creating an AST and analyzing its characteristics
\end{itemize}

With the skills you’ll acquire in this chapter, you’ll be able to build a compiler frontend for any programming language.








