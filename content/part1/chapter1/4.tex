
CMake系统使用CMakeLists.txt文件中的项目描述。顶层文件位于llvm目录“llvm/CMakeLists.txt”。其他目录也有CMakeLists.txt文件,这些文件在生成过程中递归包含。

根据项目描述中提供的信息,CMake检查安装了哪些编译器,检测库和符号,并创建构建系统文件,例如build.ninja或Makefile(取决于所选择的生成器)。也可以定义可重用的模块,例如:检测LLVM是否已安装的函数。这些脚本放在特殊的cmake目录(llvm/cmake)中,在生成过程中会自动搜索该目录。

构建过程可以通过定义CMake变量来定制。命令行选项-D用于将变量设置为一个值,这些变量在CMake脚本中使用。CMake自己定义的变量几乎总是以CMAKE\_为前缀,这些变量可以在所有项目中使用。由LLVM定义的变量以LLVM\_为前缀,但它们只能在项目定义中包含LLVM时使用。

\mySubsubsection{1.4.1.}{可定义的CMake变量}

有些变量是用环境变量的值初始化的。最值得注意的是CC和CXX,定义了用于构建的C和C++编译器。CMake尝试使用当前shell搜索路径自动定位C和C++编译器,会选择找到的第一个编译器。若本地安装了多个编译器,例如gcc和clang或不同版本的clang,那么这可能不是您构建LLVM所需的编译器。

假设使用clang17作为C编译器,clang++17作为C++编译器。可以在Unix shell中调用CMake,方法如下:

\begin{shell}
$ CC=clang17 CXX=clang++17 cmake –B build –S llvm
\end{shell}

这只会为cmake的调用设置环境变量的值。若有必要,可以为编译器可执行文件指定一个绝对路径。

CC是CMAKE\_C\_COMPILER CMAKE变量的默认值,CXX是CMAKE\_CXX\_COMPILER CMAKE变量的默认值。可以直接设置CMake变量,而不是使用环境变量。这相当于:

\begin{shell}
$ cmake –DCMAKE_C_COMPILER=clang17 \
  -DCMAKE_CXX_COMPILER=clang++17 –B build –S llvm
\end{shell}

CMake定义的其他相关变量如下所示:

% Please add the following required packages to your document preamble:
% \usepackage{longtable}
% Note: It may be necessary to compile the document several times to get a multi-page table to line up properly
\begin{longtable}{|l|l|}
\hline
\textbf{变量名} &
\textbf{功能} \\ \hline
\endfirsthead
%
\endhead
%
CMAKE\_INSTALL\_PREFIX &
\begin{tabular}[c]{@{}l@{}}
这是安装过程中每个安装路径的前缀。Unix上默认为/usr/local和\\ Windows上默认为C:\textbackslash Program Files\textbackslash{}\textless{}Project\textgreater。要在/opt/llvm目录\\ 下安装LLVM,需要设置-DCMAKE\_INSTALL\_PREFIX=/opt/llvm。\\ 二进制文件将复制到/opt/llvm/bin文件夹下,库文件将复制到\\ /opt/llvm/lib文件夹下,以此类推。
\end{tabular} \\ \hline
CMAKE\_BUILD\_TYPE &
\begin{tabular}[c]{@{}l@{}}
不同类型的构建需要不同的设置。例如,调试构建需要指定生成调\\ 试符号的选项,通常链接到系统库的调试版本。相比之下,发布版\\本使用针对库的生产版本的优化标志和链接。此变量仅用于只能处\\ 理一种构建类型的构建系统,例如:Ninja或Make。对于IDE构建系\\ 统,将生成所有配置,并且需要使用IDE在构建类型之间进行切换。\\ 取值范围如下所示: \\
\\
DEBUG:构建带有调试符号 \\
RELEASE:构建优化运行速度 \\
RELWITHDEBINFO:构建发布版本,但带有调试符号 \\
MINSIZEREL:构建会生成尺寸最小的二进制文件(也是一种优化) \\
\\
默认构建类型由CMAKE\_BUILD\_TYPE变量设置。若未设置此变\\量,则默认值取决于所使用的工具链,并且通常为空。为了生成\\ 发布版本的构建,可以设置-DCMAKE\_BUILD\_TYPE=RELEASE。
\end{tabular} \\ \hline
\begin{tabular}[c]{@{}l@{}}CMAKE\_C\_FLAGS \\ CMAKE\_CXX\_FLAGS\end{tabular} &
\begin{tabular}[c]{@{}l@{}}
这些是编译C和C++源文件时使用的编译选项,初始值取自\\ CFLAGS和CXXFLAGS环境变量。
\end{tabular} \\ \hline
CMAKE\_MODULE\_PATH &
\begin{tabular}[c]{@{}l@{}}
指定CMake模块搜索的其他目录。在默认目录之前搜索指定的目\\ 录,以分号分隔的目录列表。
\end{tabular} \\ \hline
PYTHON\_EXECUTABLE &
\begin{tabular}[c]{@{}l@{}}
若没有找到Python解释器,或者在安装了多个版本的情况下选择了\\ 错误的解释器,则可以将此变量设置为Python二进制文件的路径。\\ 这个变量只有在包含CMake的Python模块时才有效(LLVM就是这\\ 种情况)。
\end{tabular} \\ \hline
\end{longtable}

\begin{center}
表1.1 - CMake定义的其他相关变量
\end{center}

CMake为变量提供了内置帮助。{}-help-variable var选项打印var变量的帮助信息。可以输入以下命令来获取CMAKE\_BUILD\_TYPE的信息:

\begin{shell}
$ cmake --help-variable CMAKE_BUILD_TYPE
\end{shell}

也可以用下面的命令列出所有的变量:

\begin{shell}
$ cmake --help-variable-list
\end{shell}

这个清单很长,可能将输出管道传输到相应的处理程序中会更合适。

\mySubsubsection{1.4.2.}{使用LLVM定义的构建配置变量}

除了没有内置帮助之外,LLVM定义的构建配置变量的工作方式与CMake定义的相同。最有用的变量如下表所示,对首次安装LLVM的用户有用的变量和对更高级的LLVM用户有用的变量。

对于首次安装LLVM的用户有用的变量

% Please add the following required packages to your document preamble:
% \usepackage{longtable}
% Note: It may be necessary to compile the document several times to get a multi-page table to line up properly
\begin{longtable}{|l|l|}
\hline
\textbf{变量名} &
\textbf{功能} \\ \hline
\endfirsthead
%
\endhead
%
LLVM\_TARGETS\_TO\_BUILD &
\begin{tabular}[c]{@{}l@{}}
LLVM支持不同CPU架构的代码生成。默认\\ 情况下,所有这些目标都已构建。使用此变\\ 量指定要构建的目标列表,以分号分隔。目\\ 前支持的目标有AArch64、AMDGPU、\\ ARM、AVR、BPF、Hexagon、Lanai、\\ LoongArch、Mips、MSP430、NVPTX、\\ PowerPC、RISCV、Sparc、SystemZ、VE、\\ WebAssembly、X86和XCore。\\
\\
all可以作为所有目标的简写。名称区分大小\\ 写。要只启用PowerPC和SystemZ目标,可\\ 以指定-DLLVM\_TARGETS\_TO\_BUILD=\\ "PowerPC;SystemZ"。
\end{tabular} \\ \hline
LLVM\_EXPERIMENTAL\_TARGETS\_TO\_BUILD &
\begin{tabular}[c]{@{}l@{}}
除了官方支持的架构外,LLVM源代码树还\\ 包含实验目标。这些目标还在开发中,通常\\ 还不支持后端的全部功能。目前的实验得架\\ 构有ARC、CSKY、DirectX、M68k、SPIRV\\ 和Xtensa。要构建M68k目标,可以设置\\ -DLLVM\_EXPERIMENTAL\_TARGETS\_TO\\ \_BUILD=M68k。
\end{tabular} \\ \hline
LLVM\_ENABLE\_PROJECTS &
\begin{tabular}[c]{@{}l@{}}
这是要构建的项目列表,以分号分隔。项目\\ 的源代码必须与llvm目录在同一级(并排布\\ 局)。当前的列表是bolt、clang、\\ clang-tools-extra、compiler-rt、\\ cross-project-tests、libc、libclc、lld、lldb、\\ mlir、openmp、polly和pstl。\\
\\all可以用作此列表中所有项目的简写,可以\\ 在这里指定flang项目。由于一些特殊的构建\\ 需求,它还不是all列表的一部分。\\
\\
要将clang和bolt与LLVM一起构建,可以\\ 设置-DLLVM\_ENABLE\_PROJECT=\\ "clang;bolt"。
\end{tabular} \\ \hline
\end{longtable}

\begin{center}
表1.2 - 对于首次安装LLVM的用户有用的变量
\end{center}

LLVM高级用户的变量

% Please add the following required packages to your document preamble:
% \usepackage[normalem]{ulem}
% \useunder{\uline}{\ul}{}
% \usepackage{longtable}
% Note: It may be necessary to compile the document several times to get a multi-page table to line up properly
\begin{longtable}{|l|l|}
\hline
\textbf{变量名} &
\textbf{功能} \\ \hline
\endfirsthead
%
\endhead
%
LLVM\_ENABLE\_ASSERTIONS &
\begin{tabular}[c]{@{}l@{}}
若设置为ON,则启用断言检查。这些检查有助于发现错\\ 误,在开发过程中非常有用。对于DEBUG版本,默认值\\ 为ON,否则为OFF。要打开断言检查(例如,对于\\ RELEASE版本),可以设置\\ -DLLVM\_ENABLE\_ASSERTIONS=ON。
\end{tabular} \\ \hline
LLVM\_ENABLE\_EXPENSIVE\_CHECKS &
\begin{tabular}[c]{@{}l@{}}
这将启用一些大开销的检查,这些检查可能会减慢编译速\\ 度或消耗大量内存,默认值为OFF。要打开这些检查,可\\ 以指定-DLLVM\_ENABLE\_EXPENSIVE\_CHECKS=ON。
\end{tabular} \\ \hline
LLVM\_APPEND\_VC\_REV &
\begin{tabular}[c]{@{}l@{}}
若使用-version命令行选项,llc等LLVM工具除了显示其\\ 他信息外,还会显示它们所基于的LLVM版本。该版本\\ 信息基于C代码中的LLVM\_REVISION宏。默认情况下,\\ 不仅LLVM版本,而且当前Git哈希值也是版本信息的一\\ 部分。若正在跟踪主分支的开发,就能清楚地表明该工具\\ 基于哪个Git提交。若不需要,则可以使用\\ -DLLVM\_APPEND\_VC\_REV=OFF关闭此功能。
\end{tabular} \\ \hline
LLVM\_ENABLE\_THREADS &
\begin{tabular}[c]{@{}l@{}}
若检测到线程库(通常是pthread库),LLVM会自动包含线\\ 程支持。在这种情况下,LLVM会假定编译器支持TLS\\ (线程本地存储)。若不想要线程支持或者本地编译器不支\\ 持TLS,那么可以设置-DLLVM\_ENABLE\_THREADS=OFF\\ 关闭该功能。
\end{tabular} \\ \hline
LLVM\_ENABLE\_EH &
\begin{tabular}[c]{@{}l@{}}
LLVM项目不使用C++异常处理,因此在默认情况下关闭\\ 异常支持。此设置可能与项目所链接的其他库不兼容。若\\ 需要,可以通过指定-DLLVM\_ENABLE\_EH=ON来启用异\\ 常支持。
\end{tabular} \\ \hline
LLVM\_ENABLE\_RTTI &
\begin{tabular}[c]{@{}l@{}}
LLVM使用一个轻量级的自构建系统来获取运行时类型信\\ 息,C++ RTTI的生成在默认情况下是关闭的。与异常处理\\ 支持一样,这可能与其他库不兼容。要开启C++ RTTI的生\\ 成,可以设置-DLLVM\_ENABLE\_RTTI=ON。
\end{tabular} \\ \hline
LLVM\_ENABLE\_WARNINGS &
\begin{tabular}[c]{@{}l@{}}
可能的话,编译LLVM应该不会生成任何警告消息,所以\\ 打印警告消息的选项在默认情况下是打开的。要关闭它,\\ 可以设置-DLLVM\_ENABLE\_WARNINGS=OFF。
\end{tabular} \\ \hline
LLVM\_ENABLE\_PEDANTIC &
\begin{tabular}[c]{@{}l@{}}
LLVM源代码应符合C/C++语言标准,默认情况下启用对\\ 源代码进行严格检查。若可能的话,还会禁用特定于编译\\ 器的扩展。要关闭此设置,可以设置\\ -DLLVM\_ENABLE\_PEDANTIC=OFF。
\end{tabular} \\ \hline
LLVM\_ENABLE\_WERROR &
\begin{tabular}[c]{@{}l@{}}
若设置为ON,那么所有警告都被视为错误——发现警告,\\ 编译就会中止。这有助于在源代码中找到所有剩余的警告。\\ 默认情况下,它是关闭的。要打开它,可以设置\\ -DLLVM\_ENABLE\_WERROR=ON。
\end{tabular} \\ \hline
LLVM\_OPTIMIZED\_TABLEGEN &
\begin{tabular}[c]{@{}l@{}}
通常,tablegen工具与LLVM的其他部分使用相同的选项\\ 构建。同时,tablegen是用来生成大部分代码的生成器。\\ 因此,tablegen在调试构建中要慢得多,显著增加了编译\\ 时间。若将此选项设置为ON,则即使在调试构建中,\\ tablegen也会在编译时打开优化,这可能会减少编译时间。\\ 默认为OFF。要打开它,可以指定\\ -DLLVM\_OPTIMIZED\_TABLEGEN=ON。
\end{tabular} \\ \hline
LLVM\_USE\_SPLIT\_DWARF &
\begin{tabular}[c]{@{}l@{}}
若构建编译器是gcc或clang,那么打开该选项将指示编译\\ 器在单独的文件中生成DWARF调试信息。目标文件大小\\ 的减小,减少了调试构建的链接时间,默认为OFF。要打\\ 开它,可以设置-DLLVM\_USE\_SPLIT\_DWARF=ON。
\end{tabular} \\ \hline
\end{longtable}

\begin{center}
表1.3 - LLVM高级用户的变量
\end{center}

\begin{myNotic}{Note}
LLVM定义了更多的CMake变量,可以在关于CMake的LLVM文档\url{https://releases.llvm.org/17.0.1/docs/CMake.html#llvm-specific-variables}中找到完整的列表。前面的列表只包含最可能需要的部分。
\end{myNotic}
















