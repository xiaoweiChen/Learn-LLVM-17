Compiler technology is a well-studied field of computer science. The high-level task is to translate a source language into machine code. Typically, this task is divided into three parts, the frontend, the middle end, and the backend. The frontend deals mainly with the source language, while the middle end performs transformation to improve the code and the backend is responsible for the generation of machine code. Since the LLVM core libraries provide the middle end and the backend, we will focus on the frontend within this chapter.

In this chapter, you will cover the following sections and topics:

\begin{itemize}
\item 
Building blocks of a compiler, in which you will learn about the components typically found in a compiler

\item 
An arithmetic expression language, which will introduce you to an example language and show how grammar is used to define a language

\item 
Lexical analysis, which discusses how to implement a lexer for the language

\item 
Syntactical analysis, which covers the construction of a parser from the grammar

\item 
Semantic analysis, in which you will learn how a semantic check can be implemented

\item 
Code generation with the LLVM backend, which discusses how to interface with the LLVM backend and glue all the preceding phases together to create a complete compiler
\end{itemize}












